\section{Conclusion}
Programming for Linux using the C language has been a lot of fun. We have learned about the difference between programming the user space and programming the more limited kernel space. Programming a device driver were a challenge to start with, but manageable after looking at the suggested documentation such as the LDD. We were suprised to see that the way we updated the screen had the biggest impact on energy consumption. We were also suprised seing how little energy efficiency we gained from enabling tickless idle. Nonetheless, every little helps. 

\subsection{Exercise focus}
This exercise opens up for a lot of things that can be done. We have decided to put our effort in writing a rigid and correct driver that follows the newest Linux standards and conventions. In addition to this, we have put effort in reducing the energy consumption of our system. We have decided to not spend much time on developing a fancy game, and hope to get rewarded for the driver software and energy effectiveness elaboration. 

\subsection{Why we did not use DMA}
Using the DMA to write to the frame buffer could be a way of improving the energy efficiency of our game. We concluded, under Benjamin's supervision, that there was no gain in doing this. This is because the game is interactive, and and we do not know in advance whether to move a player or not. In other words, on each clock tick, there is a numerous states that the game can be in, each giving a different output to the screen.
