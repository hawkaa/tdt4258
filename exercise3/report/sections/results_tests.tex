\section{Results and Tests}

\subsection{Further development}
\paragraph{Memory Mapping}
Memory mapping is a technique used to transfer data between kernel space and userspace without copying. It is the fastest way to handle larger amount of data. The screen module uses this technique when writing to the framebuffer. We thought our driver could benefit from offering a way to use mmap to read the values of the buttons pressed. We attempted to create open and close functions for mmap and assign them to their respective fields in the \emph{vm\_operations} struct. Also a \emph{fault} function where created, wich seems to be the inheritor after the \emph{nopage} got removed. Lastly a function to assign the file data to the virtual memory data where created and assigned to the \emph{mmap} field of the \emph{file\_operations} struct. We failed however to get this to work (, wich is why it is described here and not in the \emph{description of methology} section). It seemed like the \emph{mmap} fuction never got called. Benjamin, who also thought it would be nice of our driver to support memory mapping, where helpfull but unsucessfull when trying to find the problem with us. He did however come with some suggestions where to look for clues, such as in the source code of the framebuffer. As it was not part of the assignment, and since we felt that it had allready taken to much of our time, we did not proceed with any further investigations at this point.
