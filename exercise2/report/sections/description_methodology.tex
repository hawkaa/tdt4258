\section{Description and Methodology}
Like last time, we initially developed a simple application with all the basics set up. The dac, timer and gpio was set up in the files dac.c, timer.c and gpio.c. The interrupts where handled in the interrupt\_handlers.c file. The main part of our program is the sampler.c. The interrupt handlers send and receive data to and from the sampler.c file. The gpiohandler set one of 9 modes in sampler.c and the timerhandler pulls information from the samper program and feeds it to the dac. \\
All functionality that generates sound is found in the sampler.c file. To create sound, we had one variable increase \samplerPullsPerSecound times per secound. When this variable reached a threshold, it would be set to zero. This variable would the be sent to the dac, wich would  then produce a constant tone. TODO: Skriv noe om hvordan pitch ble kontorollert. \\

To be able to play sound sequences, we created a struct to represent  a sample. This struct contained fields for duration and frequenzy. To represent a sequence, we placed several different samples in an array. Then in the sampler\_get() function, a counter was set to increase each call. When the counter had been increased enough to call it a millisecound, the duration of the current playing sample decreased. This was because the  unit of the duration field in our sample struct where set to millisecounds.
Now, The system supported sound-sequences, but there where no support for sound sequences with multiple tracks. To support multiple tracks, we figured the program should be able to play several samples in paralell. The one dimensional array containing samples was augmented to a two dimensional array. Also the variables containing the threshold and the remaining time for the current frequency was set to arrays. Now, all tracks would be updated every millisecound. To play different samples over each other, the different thresholds where scaled, added together and scaled once more before being sendt to the dac.


