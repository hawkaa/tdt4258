\documentclass{article}

\usepackage{fancyhdr} % Required for custom headers
\usepackage[utf8]{inputenc}
\usepackage{lastpage} % Required to determine the last page for the footer
\usepackage{extramarks} % Required for headers and footers
\usepackage[usenames,dvipsnames]{color} % Required for custom colors
\usepackage{graphicx} % Required to insert images
\usepackage{listings} % Required for insertion of code
\usepackage{courier} % Required for the courier font
\usepackage{lipsum} % Used for inserting dummy 'Lorem ipsum' text into the template

% Margins
\topmargin=-0.45in
\evensidemargin=0in
\oddsidemargin=0in
\textwidth=6.5in
\textheight=9.0in
\headsep=0.25in

\linespread{1.1} % Line spacing

% Set up the header and footer
\pagestyle{fancy}

\lhead{\hmwkAuthorName} % Top left header
\chead{\hmwkClass:\hmwkTitle} % Top center head
\rhead{\firstxmark} % Top right header
\lfoot{\lastxmark} % Bottom left footer
\cfoot{} % Bottom center footer
\rfoot{Page\ \thepage\ of\ \protect\pageref{LastPage}} % Bottom right footer
\renewcommand\headrulewidth{0.4pt} % Size of the header rule
\renewcommand\footrulewidth{0.4pt} % Size of the footer rule

\setlength\parindent{0pt} % Removes all indentation from paragraphs


% Document data

\newcommand{\exerciseTitle}{Exericse 1} % Assignment title
\newcommand{\exerciseClass}{TDT4258} % Course/class
\newcommand{\exerciseGroup}{Group 1} % Your name
\newcommand{\exerciseGroupMembers}{Kim Rune Solstad, Sindre Magnussen and Håkon Åmdal}

%----------------------------------------------------------------------------------------
%	TITLE PAGE
%----------------------------------------------------------------------------------------

\title{
\vspace{2in}
\textmd{\textbf{\hmwkClass:\ \hmwkTitle}}\\
\normalsize\vspace{0.1in}\small{Due\ on\ \hmwkDueDate}\\
\vspace{3in}
}

\author{\textbf{\hmwkAuthorName}}
\date{} % Insert date here if you want it to appear below your name

%----------------------------------------------------------------------------------------

\begin{document}

\maketitle
\newpage

\section{Problem 1, Regular laguages}
\subsection{Subtask A}
\includegraphics[width=0.75\columnwidth]{img/P1A.jpg} % Example image

\subsection{Subtask B}
\includegraphics[width=0.75\columnwidth]{img/P1B.jpg} % Example image

\subsection{Subtask C}
This flex program does not handle nested loops. The bracket of the innermost loop will close all of the while loops. I believe there is a way to make the program work, but I do not know how to do that.

\section{Problem 2, Grammars}
\subsection{Subtask A}
An ambigous grammar means that different derivations of the same character stream may produce different parse trees.

\subsection{Subtask B}
Yes, the grammar is ambigouos. It produces two different parse trees with leftmost derivation:\\
\includegraphics[width=0.75\columnwidth]{img/P2B.jpg} % Example image

\subsection{Subtask C}
\label{subsec:p2c}
A grammar for a context-free language is left-recursive if there exists a non-terminal symbol A that can be put through the production rules to produce a string with A as the leftmost symbol.\footnote{Notes on Formal Language Theory and Parsing, James Power, Department of Computer Science National University of Ireland, Maynooth Maynooth, Co. Kildare, Ireland.}

\subsection{Subtask D}
Yes, according to the definition in subtask C, there exists a non-terminal symbol (S::= Sp rule) that can produce a string with the symbol as the leftmost.
\end{document}
