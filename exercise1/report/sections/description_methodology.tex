\section{Description and Methodology}
\label{section:description_and_methodology}
To solve this task, we started out with a very simple solution, and then incrementally improving the energy consumption of the system. For these steps we used a basic procedure for controlling the lights; we copied the input registers into the output ones, making each button correspond to each LED output. This approach was chosen because it is easy to test the energy consumption. After some relevant testing and discussion of this solution, we implemented a somewhat more advanced LED and button behaviour, to practise our ARM assembly skills and see if the added behaviour required us to change operational mode. 

	\paragraph{Tools and utilities}
	We used the GNU toolchain for this project, e.g. \emph{arm-none-eabi-as} assembler and the \emph{arm-none-eabi-ld} linker. In addition to this, the \emph{arm-none-eabi-objcopy} was used on the binary to remove metadata. This process was automated through the exercise supplied Makefile available through itslearning. To debug the code, we used \emph{arm-none-eabi-gdb} together with Emacs. To flash the binaries to the development board we used \emph{energyAware Commander} and to test the power consumption we used the \emph{energyAware Profiler}, both provided by \emph{Silicon Labs}.
	\subsection{Setting up the GPIO}
	\label{subsection:gpio_setup}
	The gamepad created for this course requires some setup to be able to register input (button press) and create output (LED lights). We followed the instructions available at \cite[p. 24]{compendium}:
	
	\begin{itemize}
		\item Enabled the GPIO clock in the Clock Management Unit (CMU) by setting the CMU\_HFPERCLKEN0\_GPIO bit on the address pointed to by the CMU\_HFPERCLKEN0 constant.
		\item Enabled HIGH drive strength to the LEDs by setting the Port Control Register to 2 \cite[p. 766]{reference_manual}. Writing to this register let us select among STANDARD, LOWEST, HIGH, LOW settings to control the intensity of the LED light.
		\item Enabled LED output on bits 8-15 by writing $0x55555555$ to the \emph{PGIO\_PA\_CTRL} register.
		\item Enabled GPIO pins 0-7 as input (buttons) by writing $0x33333333$ to the \emph{GPIO\_PC\_MODEL} register.
		\item Enabled the internal pull-up resistors by writing 256 to \emph{GPIO\_PC\_DOUT}. This is needed to be able to read 3.3 volts from the input pins while a button is not pressed down. 
	\end{itemize}

	\subsection{Polling}
	\label{subsection:polling}
	\begin{figure}[t]
		\lstinputlisting[frame=single]{code/polling_loop.s}
		\caption{Tight polling loop}
		\label{code:polling_loop}
	\end{figure}
	The first and most naïve way of replicating the button input to the LED output, is to let the program sit in a tight loop, better known as polling. The GPIO was set up, and the program was directed to an tight, infinite loop as shown in figure \ref{code:polling_loop}.

	\subsection{Interrupts}
	\label{subsection:interrupts}
	\begin{figure}[t]
		\lstinputlisting[frame=single]{code/interrupt_handler.s}
		\caption{Interrupt handler}
		\label{code:interrupt_handler}
	\end{figure}
	There is no need to continiously read from and write to the GPIO registers. The Giant Gecko EFM32 have interrupt support, which makes us able to trigger a subroutine when a button is pressed. We followed the instructions available at \cite[p. 24]{compendium}:
	\begin{itemize}
		\item Wrote 0x22222222 to \emph{GPIO\_EXTIPSELL}, which enables interrupts on pin 0-7 from PORT C (buttons) \cite[p. 770]{reference_manual}
		\item Enabled low-to-high and high-to-low transition interrupts by writing 256 to \emph{GPIO\_EXTIFALL} and \emph{GPIO\_EXTRISE}.
		\item Enabled interrupt generation by writing $0xff$ to \emph{GPIO\_IEN} and interrupt handling by writing $0x802$ ($0x802$ corresponds to bits 1 and 11 for odd and even GPIO interrupts) to \emph{ISER0}.
	\end{itemize}
	
	


In addition to this, we had to create the interrupt handler itself and make entries to this handler at the Nested Vector Interrupt Controller (NVIC). The interrupt handler is very similar to the tight loop described in section \ref{subsection:polling}. The only difference is that the interrupt flags are cleared and the handler jumps back to where it was before the interrupt started. The main differences is shown in figure \ref{code:interrupt_handler}. The reset handler (main procedure) was set to a tight loop (using the "b ." instruction) after the interrupts was set up.

	\subsection{Energy modes}
	The EFM32 supports five different opreational modes, which is known as Energy Modes. They are different levels of functionality and power consumption, numbered EM0 to EM4. EM0 has everything turned on, thus higher energy consumption. The higher the level, the less functionality and power consumption you get. Please see \cite[p. 2]{energy_optimization_application_note} for more details. 

	\subsubsection{EM1}
	\label{subsubsection:em1}

	\begin{figure}[t]
		\lstinputlisting[frame=single]{code/em1.s}
		\caption{Sleep loop for Energy mode 1}
		\label{code:em1}
	\end{figure}
	The system is put to sleep (Energy mode 1) using the \emph{wfi} instruction. The system does not automatically go back to sleep after an interrupt is done, so we had to write a small loop which will continiously call the \emph{wfi} instruction. The system is now in an infinite loop, but it is not continiously running, it only does one iteration each time an interrupt handler is complete. See figure \ref{code:em1}.

	\subsubsection{EM2}
	\label{subsubsection:em2}
	
%	\begin{figure}[t]
%		\lstinputlisting[frame=single]{code/wfi_scr.s}
%		\caption{Entering Energy mode 2}
%		\label{code:wfi_scr}
%	\end{figure}

	To avoid looping, we can put the system to EM2. In addition to this, we  wanted to make the system automatically go back to sleep after the interrupt handlers were done. We did both of these things by setting the value of the System Control Register (SCR) to the constant $6$. \emph{wfi} is still needed to put the system to sleep, but no loop is needed.

		\subsubsection{EM3}
		\label{subsubsection:em3}	
		%\begin{figure}[t]
		%	\lstinputlisting[frame=single]{code/em3.s}
		%	\caption{Enabling Energy mode 3}
		%	\label{code:em3}
		%\end{figure}

		To improve the energy efficiency of our system even more, we tried out Energy Mode 3. To do enable this operational mode, we had to turn off the low frequency oscilators. This was done by writing the constant $0$ to the CMU\_LFCLKSEL.
	
		\subsubsection{EM4}
		\label{subsubsection:em4}
	
		\begin{figure}[t]
			\lstinputlisting[frame=single]{code/em4.s}
			\caption{Enabling Energy mode 4}
			\label{code:em4}
		\end{figure}

		Energy mode 4 is the highest level, and thus the mode with the least functionality. We decided to try out whether this mode was sufficient for the LED functionality. To get the system to EM4 was a little more complicated than the previous Energy Modes. In order to put our system in this state, we had to write a sequence of bytes to the EMU\_CTRL (see \cite[p.~112]{reference_manual} for more details). Please see figure \ref{code:em4} for assembly code.

	
	\subsection{Improving functionality - Animated LED rotation}
	Since the assembly code so far was rather simple, and mostly consisted of writing predefined constants to various registers, we wanted to make a little more advanced code. We decided to animate the LEDs by doing a rotate operation at a given time interval.
	
\begin{figure}[t]
		\lstinputlisting[frame=single]{code/systick_enable.s}
		\caption{Enabling SysTick-exceptions}
		\label{code:systick_enable}
\end{figure}

	\subsubsection{SysTick handler}
	To animate the LEDs, we needed a timer that triggered an interrupt at a predefined interval. This was done by using SysTick-exceptions. We found information how to do this in \cite[section 4.4]{cortex_m3_ref_man}. This is the procedure we followed to enable SysTick exceptions:
	\begin{itemize}
	\item Enable the SysTick-timer setting the enable bits high in SysTick control and status register, CTRL. SysTick CTRL is given as \emph{st\_ctrl} in figure \ref{code:systick_enable}.
	\item Setting the count-down timer to the longest possible interval time (0x00ffffff) to SysTick reload value register, LOAD. This value is loaded each time the SysTick-timer is done counting and starts a new iteration. SysTick LOAD is written as \emph{st\_load} in figure \ref{code:systick_enable}.
	\end{itemize}
	
To make the controller act correctly, we changed the Exception-vector table so that we don't get stuck in the \emph{dummy\_handler} routine. Instead we wanted the program to jump to our own \emph{systick\_handler} routine, which rotates the LED lights.

	\subsubsection{Rotating}
	Our program rotates the lights left to right. To make the LEDs rotate correctly when we have reached the rigthmost LED, we had to do a little trick. Since writing to the pins is done with a 16-bit word and we only wanted the eight most significant bits set, we moved the light around to the leftmost LED by doing a small number of bit operations. These operations are best explained in an example (Note the binary numbers is inverted compared to the LEDs):
	\begin{itemize}
	\item The initial value of the pins is $0111111100000000$ in binary, which means that the rightmost LED is active. On the next tick we need to rotate the left-most zero around (making the light appear on the left).
	\item The tick occurs. The value of the pins is or-ed with 0000000011111111, and we get 0111111111111111. This makes sure we by default shift in a LED that is turned off.
	\item  In order to find out that the left-most bit is zero, we applied a logical and operation with the bitstring 100000000000000. If the result of this was zero, the ninth bit was forced to zero by applying an and-operation to the following bit string 1111111101111111, which in this example is true.
	\item Then we do a logical shift left to move the LED right.
	\end{itemize}

We used this test case to generate the code for the SysTick handler. Check out ex1.s for the implementation.
