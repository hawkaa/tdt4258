\section{Introduction}
In this exercise, we were supposed to implement a system that let the user control the LEDs in some way by pressing the gamepad buttons. We were also required to implement the functionality by using an interrupt routine. During the solving of this task, we were supposed to analyze the power consumption and correctness of the system. The learning outcome of this task was (as stated in \cite[p. 19]{compendium}):
\begin{itemize}
	\item Get to know the \boardName, the GPIO and the ARM Cortex-M3 arcitechture.
	\item Get to know the GNU toolchain, object files and the task of a linker.
	\item Programming in assembly for the ARM Cortex-M3.
	\item Measuring and optimizing power consumption.
\end{itemize}
We approached the task by first creating a simple program which we graudually improved the energy efficiency of by introducing interrupts and various operational modes. Then we implemented a more advanced program to practise our assembly skills and see if the added functionality required us to change operational mode. During the development, we tested for correctness and power consumption continiously, which maximized the learning outcome of this task.

