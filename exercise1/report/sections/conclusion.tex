\section{Conclusion}
This report shows how to use the \boardName, the GPIO and the ARM Cortex-M3 architecture by programming in assembly. It also shows some simple techniques for reducing the energy consumption of the system by itroducing different operational modes (better known as Energy Modes). \\
\\
The more functionality you want to have on your \boardName ~chip, the lower Energy Mode you will have to use, thus higher power consumption. Although $70\mu A$ power consumption in EM4 is rather amazing, it does no good when the system does not wake up from GPIO interrupts. For more advanced programs, like the ones that use the SysTick functionality, you will have to go as low as EM1 in order to have the timer tick interrupts generated.\\
\\
A noteworthy result, is the fact that when enabling interrupts, the power consumption was slightly reduced, even though the program was spinning in a tight loop. This is very likely to be power consumed by the bus and units for memory access. With the polling code, the program constantly read and wrote to the GPIO register. Since the GPIO is memory mapped, the system will then use the memory bus on every cycle. This is not the case with the interrupt enabled code, having the chip only access the GPIO registers when needed.

